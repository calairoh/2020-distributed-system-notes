\subsection{Run-time architectures}

    \subsubsection{Client-Server}
    It's one of the most common architecture today.

    The main caracteristic is that the components have different roles, in other words: 
    \begin{itemize}
        \item Servers provide a set of services through a defined API, and most important, those API are \textbf{passive}, which meand that the server is waiting for client calls.
        \item Users access those services through clients.
        \item The communication is message based.
    \end{itemize}

    The typical architecture is the \textbf{two tiers architecture}, which is basically
    composed by a server and a client and, based on the component load, we can distingue between
    \textit{thick-client} and \textit{thin-client}.\\
    Moreover, in this architecture the server can also operate like a client, 
    for example calling another server to get some information 
    \textit{(e.g. calling a DB server in order to get some kind of data)}. 
    In this case we speak about \textbf{three-tiered client-server architecture}. 
    Obviously this concept can be extended to \textit{n} tiers, so we can have \textbf{multi-tiered client-server applications}.
    Those applications can be classified looking at the way such services are assigned to the different tiers.

    \subsubsection{Service Oriented}
    In this architecture, the service became the most important thing.\\
    We basically 4 concepts:
    \begin{itemize}
        \item Services: represent loosely coupled units of functionality
        \item Service providers: entities which exported the services
        \item Brokers: entities which hold the description of available 
                services to be searched by interested consumers and reduce the
                dependency between client and server
        \item Service consumers: entities which bind and invoke the services
                they need and exposed by the Brokers
    \end{itemize}
    In the end, the \textit{Orchestration} is the process of invoking a set
    of services in an ad-hoc workflow to satisfy a specific goal.\\
    In the practice, there are many ways to implement this architecture.
    The most commons are:
    \begin{itemize}
        \item OGSI (Open Grid Services Infrastructure)
        \item JXTA
        \item Jini
        \item Web Services
    \end{itemize}
    
    \paragraph{Web Services}
    \subparagraph{Def}
    A software system designed to support interoperable machine-to-machine
    interaction over a network.\\
    Related on the Service Oriented Architecture, we can identify
    the Web Services as one of the most important application.\\
    \textbf{Note that:} it exposes an interface that is described by \textit{WSDL},
    which stands for \textit{Web Service Description Language}, and it
    contains the set of operations exported by the web service.\\
    Web service \textbf{operations} are invoked through \textit{SOAP}
    \subparagraph{SOAP}
    A protocol, based on \textit{XML}, which defines the way messages
    (operation calls) are actually exchanged. It's usually based on \textit{HTTP}
    but other transport protocols can be used.
    \subparagraph{UDDI}
    It stands for \textit{Universal Description Discovery And Integration}
    describes the rules that allows web services to be exported and searched
    through a \textit{registry}

    \paragraph{REST}
    \subparagraph{Def}
    \textit{REST} stands for \textit{REpresentational State Transfer}\\
    \\
    This style is very important for two main reasons:
    \begin{itemize}
        \item It's a nice way to describe the web
        \item It's a set of principles that defin how Web standards
                are supposed to be used
    \end{itemize}
    The REST style is an optimal solution in order to have:
    \begin{itemize}
        \item Scalability of component interactions
        \item Generality of interfaces
        \item Indipendent deployment of components
        \item Intermediary components to reduce latency, enforce
            security and encapsulate legacy systems
    \end{itemize}
    The main contraints are:
    \begin{itemize}
        \item Interactions are client-server
        \item Interactions are stateless, otherwise:
        \begin{itemize}
                \item If it's stateful there is a big dependency between client
                        and server
                \item A client can't call another server, because it needs that
                        specific server, because it has the interactions story 
        \end{itemize}
            Moreover, with the stateless contraint:
        \begin{itemize}
            \item The system becames very scalable
            \item I can introduce a caching service, in order to improve performance
        \end{itemize} 
        \item The data within a response to a request must be implicitly or explicitly
            labeled as cacheable or non-cacheable
        \item Each component cannot "see" beyond the immediate layer with which they
            are interacting, so we can say that the \textbf{REST is layered}.
        \item Components expose a uniform interface
        \item Clients must support \textit{code-on-demand}, but note that this is an 
            \textsl{optional constraint}
    \end{itemize}
    
    \subparagraph{REST: Uniform interface contraints}
    The uniform interface exposed by components must satisfy four contraints:
    \begin{itemize}
        \item \textbf{Identification of resources:} each resource must have an Identification
            (usually an \textit{URI}) and everything that have an ID is a valice resource 
            (including a service)
        \item \textbf{Manipulation of resources through representations}
        \begin{itemize}
            \item \textit{REST} components communicate by transferring a representation of a
                resource in a format matching one of an evolving set of standards data types,
                selected dinamically based on the capabilities or desires of the recipient and
                the nature of the resource
            \item Whether the representation is in the same format as the raw
                resource, or is derived from the resource, remains hidden behind the interface
            \item A representation consists of data and metadata describing the data
        \end{itemize}
        \item \textbf{Self-descriprive messages}
        \begin{itemize}
            \item Control data defines the purpose of a message between components, such as the 
                action being requested or the meaning of a response
            \item It is also used to parameterize requests and override the default behavior of
                some connecting elements
        \end{itemize}
        \item \textbf{Hypermedia as the engine of application site}
            Clients move from a state to another each time process a new representaiton, usually
            linked to other representation through hypermedia links
    \end{itemize}

    \subsubsection{Peer-to-peer}
    In a \textit{peer-to-peer} applications all components play the same role, so there is no 
    distinction between clients and servers.\\
    But, why the \textit{peer-to-peer} model is born? The main reasons are:
    \begin{itemize}
        \item Client-server does not scale well, due to centralization of service provision and management
        \item The server is also a single point of failure
        \item P2P leverages off the increased availability of broadband connectivity and processing
            power at the end-host to overcome such limitations
    \end{itemize}
    Moreover, \textit{P2P} promotes the sharing of resources and services through direct exchange between peers.\\
    Resources can be:
    \begin{itemize}
        \item Processing cycles
        \item Collaborative work (\textit{i.e. Skype})
        \item Storage space
        \item Network bandwidth (\textit{i.e. ad hoc networking, internet})
        \item Data
    \end{itemize}

    \subsubsection{Object-Oriented}
    The distributed components encapsulate a data structure providing an API to access
    and modify it
    \begin{itemize}
        \item Each component is responsible for ensuring the integrity of the data structure
            it encapsulate
        \item The internal organization of such data structure id hidden to the other components
            (who may access it only through the API mentioned above)
    \end{itemize}
    Moreover, components interact through \textit{RPC} and we can consider it as a \textit{P2P} model,
    but it's often used to implement client-server application.\\
    This architecture has many advantages:
    \begin{itemize}
        \item Information hiding hides complexity in accessing/managing the shared data
        \item Encapsulation plus information hiding reduce the management complexity, in fact we
            can provide some load-balancing tecniques in order to share the server load.
        \item Objects are easy to reuse among different applications
        \item Legacy components can be wrapped within objects and easily integrated in new applications
    \end{itemize}

    \subsubsection{Data-Centered}
    The main principle is that components communicate through a common (usually passive) repository.
    On the repository, data can be added to the repository or taken from it.\\
    Access to the repository is usually synchronized and the communication with the 
    repository is usually through RPC.

    \paragraph{Linda}
    \textit{Linda} is one of the possible example of Data-Centered model.\\
    The main characteristics are:
    \begin{itemize}
        \item Communication is persistent, implicit, content-based, generative
        \item High degree of decoupling
        \item Data is contained i ordered sequences of typed fields (\textit{tuples})
        \item Tuples are stored in a persistent and global shared space (\textit{tuple space})
    \end{itemize}

    \paragraph{Architectural issues}
    \subparagraph{Scalability}
    The main problem of the data-centered architectures is they're cannot be easily scaled
    on a wide-area, mainly because it's tricky to store and replicate tuples efficiently in 
    such a wide area. Another problem is the routing. For the same reasons routing queries
    efficiently in very large system is not a simple problem to approach.
    \subparagraph{Proactive}
    The model is proactive in the sense that processes explicity request a tuple query,
    and so there's is no way to be notified where an info is available. In order to have a
    reactive and asynchronous behavior must implemented with an extra process and a
    blocking operation.
    \subparagraph{Bottleneck}
    In general the repository is a bottleneck for the performance

    In the end, in the commercial implementations, only client access to a server holding the
    tuple space, and, in order to add reactivity to the system, in general we add reactive
    primitives, such as \textit{notify}.

    \subsubsection{Event-Based}
    The main idea is that components collaborate by exchanging information about occurrent \textit{events}.
    In particual we distingush among:
    \begin{itemize}
        \item \textit{publishers}: components which publish notifications about the events they observe
        \item \textit{subscribers}: components which subscribe to events they are interested to be notified about
    \end{itemize}
    In particular, the communication is:
    \begin{itemize}
        \item Purely message based
        \item Asynchronous
        \item Multicast
        \item Implicit
        \item Anonymous
    \end{itemize}

    \subsubsection{Mobile code}
    It's based on the ability of relocating the components of a distributed application at run-time.
    In same case you can also move the entire state and not only the code.
    We can have multiple paradigms:
    \begin{itemize}
        \item Client-Server
        \item Remote evalutation
        \item Code on demand
        \item Mobile agent
    \end{itemize}
    The mobility can be divided in two types:
    \begin{itemize}
        \item \textit{Strong mobility} is the ability of a system to allow migreation of both code and
            execution state
        \item \textit{Weak mobility} is the ability of a system to allow code movement across different
            computational enviroments (\textit{e.g. JavaScript})
    \end{itemize}
    This solution provides a great flexibility for add, upgrade and enrich services at run-time, but, 
    at the same time, securing mobile code applications is tricky.
    
    \paragraph{CREST}
    \textit{CREST} stands for \textit{Computational REST} and it joins together the concepts of \textit{REST}
    with mobile code, so, instead of \textit{representations}, interacting parties exchanges \textit{computations}.

    \subsection{The interaction model}
    \subsubsection{Distributed algorithms}
    A traditional algorithm is defined as a sequence of steps, in which the process
    execution speed is the only variable that influence performance. In a distributed
    system we have also the concept of the distributed algorithm, which is difference 
    from the traditional one, in the sense that the performance is also influenced by 
    the transmission of messages between them. In general the behavior of a distributed
    algorithm is influenced by:
    \begin{itemize}
        \item The rate at which each process proceeds
        \item The performance of the communication channels
        \item The different clock drift rates
    \end{itemize}
    We can distinguish between:
    \begin{itemize}
        Synchronous DS: there are lower and upper bounds
        Asynchronous DS: there are no bounds
    \end{itemize}
    \subparagraph{Caution}
    Any solution that is valid for an asynchronous DS is also valid for a synchronous one